\documentclass{article}

\usepackage[english]{babel}
\usepackage[a4paper, top=2cm, bottom=2cm, left=3cm, right=3cm, marginparwidth=1.75cm]{geometry}
\usepackage{graphicx}
\usepackage{hyperref}

\title{The Orange Cat Gene}
\author{Anamaria Hodivoianu}

\begin{document}
\maketitle

\section{Introduction}
The gene that gives orange cats their fur color has been a mistery for a long time. And while orange cats are mostly male, calico cats and tortoiseshell cats are mostly female. This difference has long been fascinating scientists, but it was only recently that the mutation responsbile for this has been discovered. The discovery was made by not one, but two independent teams of researchers, one from Stanford University and one from Kyushu University.

\section{Calico and Tortoiseshell Cats}
Calico cats have three colors in their fur, white, black and orange, while Tortoiseshell cats have two colors in their fur, black and orange, which are grouped in patches. Such cats have parents of different colors, black and orange, and they are almost always female. Calico cats also have white fur because of an unrelated gene that causes pigment production to stop in some cells. This has led scientists to believe that the gene responsible for the orange color is located on the X chromosome. Since females have two X chromosomes, they inherit one from each parent, and they can have both the orange and the black gene. These genes are expressed in different cells, which is why the fur is patchy, because cells only need one X chromosome to be active. Since males have only one X chromosome, they can only have one of the two genes, and they are either orange or black.

\section{The Orange Gene}
Orange hair in most mammals, humans included, is caused by a protein called Mc1r. This protein is responsible for the pigment that skin cells called melanocytes produce. More Mc1r proteins will cause a dark pigment, while less will cause a lighter pigment, such as orange. There are mutations that cause the Mc1r protein to be less active, which leads to red hair. But the gene that encodes the Mc1r protein is not located in the X chromosome, and orange cats don't even have the mutation that causes red hair in other mammals.
To gather more information, a team of researchers from Stanford University, led by Dr. Greg Barsh, analyzed skin cells from both orange and non-orange cats. Specifically, they looked at the RNA that the melanocytes produced, how much, and what gene was expressed by that RNA. They found that in orange cats, melanocytes produced 13 times more RNA that encoded the Arhgap36 gene than in non-orange cats. This gene is located on the X chromosome. However, there were no mutations in the gene itself, but in another part of the DNA that, while it didn't influence the protein, it was possible it influenced the cantity of RNA produced. To solidify their findings, the researchers analyzed a database of cat genomes and found this mutation in all orange, calico and tortoiseshell cats.
The same discovery was made by a team of researchers from Kyushu University, led by Dr. Hiroyuki Sasaki. They also found that in calico cats, the regions with the orange fur had more RNA that encoded the Arhgap36 gene than in the other regions. The Arhgap36 gene was known to influence embryonic development, but it was not known to influence fur color until now.

\section{Conclusion}
In conclusion, the gene that gives orange cats their fur color has been discovered. This gene is located on the X chromosome, and it encodes the Arhgap36 gene. This gene influences the amount of RNA produced by melanocytes, which in turn influences the amount of pigment produced.

\end{document}