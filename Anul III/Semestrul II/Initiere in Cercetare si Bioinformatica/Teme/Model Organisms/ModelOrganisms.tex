\documentclass{article}

\usepackage[english]{babel}
\usepackage[a4paper, top=2cm, bottom=2cm, left=3cm, right=3cm, marginparwidth=1.75cm]{geometry}
\usepackage{graphicx}
\usepackage{hyperref}

\title{Model Organisms}
\author{Anamaria Hodivoianu}

\begin{document}
\maketitle

\section{Introduction}
Model organisms are non-human species that are used to by scientists to study different biological processes in cases when experimenting on humans is not possible or not ethical. The results obtained through these experiments can also be applied to humans thanks to the similarities between the species. Many important discoveries in biology have been made with model organisms, which have significantly improved the lives of humans, but also of animals.

\section{Examples of Model Organisms}
\subsection{Escherichia coli}
Escherichia coli, or E. coli, is a bacterium that lives in the intestines of humans and other animals. It is used in molecular genetics, but also is biotechnology and microbiology. It is the most widely used organism in the study of recombinant DNA. It is easy to grow in a laboratory, and it has a short generation time, which makes it ideal for studying the effects of mutations.
\subsection{Saccharomyces cerevisiae}
Saccharomyces cerevisiae, or baker's yeast, is a eukaryotic organism that is used in the study of cell biology and genetics. It has been used in the discovery of many cell division genes, which are important for cancer research.
\subsection{Drosophila melanogaster}
Drosophila melanogaster, or the common fruit fly, is a eukaryotic organism that is used in the study of genetics and development. It has a short generation time, and it is easy to manipulate genetically. It is most well known thanks to the work of Thomas Hunt Morgan, who discovered that genes are located on chromosomes. 
\subsection{Caenorhabditis elegans}
Caenorhabditis elegans, or the roundworm, is a eukaryotic organism that is used in the study of development and neurobiology. It was the first multicellular organism to have its genome sequenced, and the only organism to have its connectome mapped. 
\subsection{Mus musculus}
Mus musculus, or the house mouse, is a eukaryotic organism that is used in the study of genetics and development. It is the most widely used mammalian model organism, and it has been used in the study of many diseases, including cancer and diabetes. It is also used in the study of behavior and cognition.
\subsection{Danio rerio}
Danio rerio, or the zebrafish, is a eukaryotic organism that is used in the study of development and toxicology. It is a popular model organism because it has a transparent embryo, which makes it easy to study development in real time.

\section{Problems}
While model organism have led to many important discoveries, there are also some problems associated with their use. Because of the differences between species, the results obtained from model organisms may not always be applicable to humans. For example, some drugs that are effective in mice may not be effective in humans. 

\section{Ethics}
Experimenting on animals is however a controversial topic, with arguments for and against it. The main argument for using model organisms that it has led to many important discoveries in medicine and biology, and that it is necessary for the development of new treatments. The arguments against animal experimentation include the fact that it is cruel and inhumane, and that it is not always necessary. Many countries have laws that regulate animal experimentation, and there are also many organizations that work to promote the ethical treatment of animals.

\section{Conclusion}
In conclusion, model organism are important for scientific research, and they have led to many important discoveries in biology and medicine. However, there are also some problems associated with their use, and the ethical implications of animal experimentation should be considered.

\end{document}