\documentclass{article}

\usepackage[romanian]{babel}
\usepackage[a4paper, top=2cm, bottom=2cm, left=3cm, right=3cm, marginparwidth=1.75cm]{geometry}

\title{Gattaca}
\author{Anamaria Hodivoianu}

\begin{document}
\maketitle

\section{Introducere}
Gattaca este un film SF din 1997 care se desfasoara intr-un viitor in care ADN-ul uman este complet secventiat, acest lucru schimband profund societatea. Filmul exploreaza implicatiile secventierii ADN-ului asupra vietii oamenilor si a societatii in general.

\section{Impactul secventierii ADN-ului}
Secventierea ADN-ului uman a avut un impact major asupra vietii oamenilor. Acest avans a permis copiilor sa fie produsi in laborator, cu o selectie genetica atenta astfel incat sa se asigure ca nu vor avea predispozitii genetice pentru diverse boli, atat fizice cat si mentale, si ca vor mosteni doar cele mai bune trasaturi genetice ale parintilor lor. Desi aceasta tehnologie pare ca aduce doar beneficii, nu toti parintii aleg sa isi conceapa copiii in acest mod, iar unii prefera sa aleaga modul natural. Acest lucru a dus la o noua ordine sociala, in care oamenii sunt impartiti in doua categorii: valizi si invalizi. Oamenii valizi sunt cei care au fost conceputi in laborator si care au fost selectati genetic, in timp ce oamenii invalizi sunt cei care s-au nascut natural si care nu au fost selectati genetic. Desi teoretic discriminarea bazata pe ADN este interzisa prin lege, este extrem de raspandita in societate. Deoarece este foarte usor sa se obtina informatii despre ADN-ul unei persoane, aceasta discriminare este foarte greu de prevenit. ADN-ul este de multe ori folosit drept interviu neoficial, oamenii valizi avand sanse mult mai mari sa isi atinga obiectivele in viata decat oamenii invalizi, care sunt de multe ori marginalizati si discriminati. In timp ce oamenilor invalizi nu li se acorda nicio sansa, oamenilor valizi li se deschid multe oportunitati doar pe baza ADN-ului. Impactul negativ asupra oamenilor invalizi este evident, insa exista si un impact negativ asupra oamenilor valizi. Cei care sunt nascuti cu un ADN foarte bun sunt pusi sub presiunea de a fi perfecti, iar orice sub perfectiune este vazut ca un esec.

\section{Dileme etice}
Desi filmul Gattaca este fictiune, acesta ridica o serie de intrebari importante despre impactul secventierii ADN-ului asupra societatii si asupra vietii oamenilor. Oamenii de stiinta sunt adeseori atat de preocupati de si fascinati de ceea ce pot face, incat nu se gandesc la consecintele etice ale posibilelor descoperiri si daca acestea ar trebui sa fie facute. Din acest motiv, este important ca oamenii de stiinta sa fie constienti de consecintele etice ale cercetarilor lor si sa ia in considerare aceste consecinte inainte de a face descoperiri care ar putea schimba societatea in moduri pe care nu le pot prevedea. Secventierea ADN-ului uman ridica o serie de dileme etice si morale, precum daca selectia genetica a copiilor este morala, posibilitatea discriminarii bazate pe ADN si daca oamenii ar fi mai bine sau nu sa stie informatii despre ADN-ul lor si ca viata lor sa fie dictata de catre aceste informatii si nu de catre ei insisi.

\section{Concluzie}
In concluzie, filmul Gattaca prezinta un posibil viitor in care secventierea ADN-ului uman a fost completata si modul in care societatea si viata de zi cu zi a oamenilor a fost schimbata de aceasta tehnologie, lasand telespectatorii sa se intrebe daca aceasta tehnologie, in cazul in care ar exista, ar fi cu adevarat benefica sau daca ar trebui sa fie folosita cu precautie. 

\end{document}